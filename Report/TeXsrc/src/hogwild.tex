\section{\hogwild}

Prior to 2011, parallel stochastic gradient methods had been introduced, but
most suffered from poor scaling due to the necessity of locks. A naive
implementation could look like:
\begin{breakablealgorithm}
  \caption{Very Naive Parallel Stochastic Gradient}
  \label{alg:naivePSG}
  \begin{algorithmic}[1]
    \Require Number of data points $N$, seperable loss function $f
    = \sum_{e \in E} f_e(x_e)$, Initial $x$.
    \For{epoch $= 0 \to$ MAX\_EPOCHS}
      \State \#pragma omp parallel for
      \For{$k = 0 \to N$}
        \State Choose $i$ uniformly from $\{1, \dots, |E|\}$.
        \State \#pragma omp critical
        \Indent
          \State Read current parameters $x$.
          \State Compute $\nabla f_i(x)$.
          \State $x \gets x - \eta \nabla f_i(x)$.
        \EndIndent
      \EndFor
    \EndFor
  \end{algorithmic}
\end{breakablealgorithm}
But it's clear here that such an algorithm would only effectively be
parallelizing the unform sample of $i$ in $\{1, \dots, |E|\}$. You can improve
the above by selectively locking the values in $x$ which actually change in the
stochastic gradient step (say if $\nabla f_i(x)$ was sparse), but because the
process of acquiring locks is much more expensive than floating point
arithmetic, this helps little.

However, in 2011, the article "\hogwild: A Lock-Free Approach to Parallelizing
Stochastic Gradient Descent" proposed a very simple solution to this problem.
Remove the locks!
\begin{breakablealgorithm}
  \caption{Very Naive Parallel Stochastic Gradient}
  \label{alg:naivePSG}
  \begin{algorithmic}[1]
    \Require Number of data points $N$, seperable loss function $f
    = \sum_{e \in E} f_e(x_e)$, Initial $x$.
    \For{epoch $= 0 \to$ MAX\_EPOCHS}
      \State \#pragma omp parallel for
      \For{$k = 0 \to N$}
        \State Choose $i$ uniformly from $\{1, \dots, |E|\}$.
        \State Read current parameters $x$.
        \State Compute $\nabla f_i(x)$.
        \State $x \gets x - \eta \nabla f_i(x)$.
      \EndFor
    \EndFor
  \end{algorithmic}
\end{breakablealgorithm}
